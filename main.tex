\documentclass[eng,printmode]{mgr}
\usepackage{polski}
\usepackage{url}
\usepackage{listings}
\usepackage[utf8]{inputenc}
\usepackage[T1]{fontenc}

\usepackage{graphicx}
\usepackage{subfigure}
\usepackage{psfrag}


\usepackage{amsmath}
\usepackage{amsfonts}

\usepackage{supertabular}
\usepackage{array}
\usepackage{tabularx}
\usepackage{hhline}

\usepackage{showlabels}

\newcommand{\R}{I\!\!R}
\newtheorem{theorem}{Twierdzenie}[section]

\title{Tytuł pracy inżynierskiej}
\engtitle{English title}
\author{Maciej Bakowicz}
\supervisor{dr hab. inż. Imię Nazwisko Prof. PWr, I-6}

\field{Teleinformatyka (TIN)}
\specialisation{Projektowanie sieci teleinformatycznych (TIP )}


\begin{document}
\bibliographystyle{plabbrv}

\maketitle 

\tableofcontents 

\chapter{Wstęp}  
Niniejsza praca dyplomowa opisuje liczne problemy związane z zagadnieniami dotyczącymi pracy w grupie takimi jak problem komunikacji, wzajemnego zrozumienia czy samej organizacji, struktury a także formy kooperacji między członkami zespołu.
\\
Pierwsza cześć składa się ze szczegółowego omówienia problemu występującego na rynku, zebraniu wymagań, przedstawienia gotowych już rozwiązań oraz porównania z proponowanym projektem.
\\
Druga cześć opisuje proces implementacji aplikacji internetowej. Zawiera min opis wykorzystanych technologii, narzędzi. Znalazł się tam również system pracy nad projektem oraz jej podział.
\\
Końcowa, trzecia część obejmuje podsumowanie całości pracy; Zawiera się w nim to co zostało zrealizowane do chwili obecnej oraz dalsze plany dotyczące rozwoju aplikacji w przyszłości wraz z propozycjami dodatkowych funkcjonalności . W tej części zawarte są również wnioski oraz konkluzje dotyczące procesu programistycznego oraz samego projektu aplikacji. \cite{Node.js}

\section{Cel pracy}
Celem pracy jest opisanie projektu aplikacji, procesów na nią się składających oraz zaimplementowanie jako działającego programu rozwiązującego problemy pracy w grupie.
\\
Projekt ma zarówno wyjaśnić istniejący problem dotykający tego typu kooperacji między ludźmi a także dostarczyć gotowe rozwiązania niwelujące zaistniałe niedogodności oraz uprościć procesy powiązane z tym zagadnieniem. Głównym zadaniem jest próba poprawy mechanizmów komunikacji, systematyzacji pracy oraz podziału zadań między poszczególnych członków zespołu w sposób łatwy, wygodny a przede wszystkim efektywny, wydajny i przynoszący zyski.
\section{Problem}
Problemy wynikające z potrzeby współpracy z innymi ludźmi spotyka się zazwyczaj w sytuacjach, gdzie realizowany jest skomplikowany, obszerny projekt, wymagający znajomości wielu obszarów powiązanych ze sobą zagadnień, gdzie należy zastosować podział pracy na mniejsze fragmenty pomiędzy wielu ludzi, którzy posiadają odpowiednią wiedzę lub doświadczenie. Im większy, bardziej złożony projekt tym potrzeba podziału obowiązków jest silniejsza. Czy to wynikająca z samej struktury projektu czy innych czynników zewnętrznych takich jak ograniczenia czasowe lub jakościowe. \\
W dalszych następstwach powstają kolejne problemy, wynikające już z samej ilości zaangażowanych do projektu osób. Aby praca zespołowa przynosiła efekty należy musi być ona dobrze zorganizowana oraz zaplanowana.
\chapter{Analiza istniejących rozwiązań}  
\section{Badanie rynku}
\section{Zastosowane technologie}
\subsection{Języki programistyczne}
\subsection{Systemy bazodanowe}
\subsection{Wykorzystane narzędzia}
\chapter{Założenia projektowe}
\chapter{Wymaganie dotyczące projektu}
\section{Wymaganie funkcjonalne}
\section{Wymagania niefunkcjonalne}
\chapter{Projekt aplikacji}
\section{Zastosowane rozwiązania}
\section{Struktura danych}
\addcontentsline{toc}{chapter}{Bibliografia}
\bibliography{bibliografia}


\end{document}
