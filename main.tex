\documentclass[eng,printmode]{mgr}
\usepackage{polski}
\usepackage{url}
\usepackage{listings}
\usepackage[utf8]{inputenc}
\usepackage[T1]{fontenc}

\usepackage{graphicx}
\usepackage{subfigure}
\usepackage{psfrag}


\usepackage{amsmath}
\usepackage{amsfonts}
\usepackage{enumitem}
\usepackage{supertabular}
\usepackage{array}
\usepackage{tabularx}
\usepackage{hhline}

\usepackage{showlabels}

\newcommand{\R}{I\!\!R}
\newtheorem{theorem}{Twierdzenie}[section]

\title{Tytuł pracy inżynierskiej}
\engtitle{English title}
\author{Maciej Bakowicz}
\supervisor{dr hab. inż. Imię Nazwisko Prof. PWr, I-6}

\field{Teleinformatyka (TIN)}
\specialisation{Projektowanie sieci teleinformatycznych (TIP )}


\begin{document}
\bibliographystyle{plabbrv}

\maketitle 

\tableofcontents 

\chapter{Wstęp}  
Niniejsza praca dyplomowa opisuje liczne problemy związane z zagadnieniami dotyczącymi pracy w grupie takimi jak problem komunikacji, wzajemnego zrozumienia czy samej organizacji, struktury a także formy kooperacji między członkami zespołu.
\\
Pierwsza cześć składa się ze szczegółowego omówienia problemu występującego na rynku, zebraniu wymagań, przedstawienia gotowych już rozwiązań oraz porównania z proponowanym projektem.
\\
Druga cześć opisuje proces implementacji aplikacji internetowej. Zawiera min opis wykorzystanych technologii, narzędzi. Znalazł się tam również system pracy nad projektem oraz jej podział.
\\
Końcowa, trzecia część obejmuje podsumowanie całości pracy; Zawiera się w nim to co zostało zrealizowane do chwili obecnej oraz dalsze plany dotyczące rozwoju aplikacji w przyszłości wraz z propozycjami dodatkowych funkcjonalności . W tej części zawarte są również wnioski oraz konkluzje dotyczące procesu programistycznego oraz samego projektu aplikacji. \cite{Node.js}

\section{Cel pracy}
Celem pracy jest opisanie projektu aplikacji, procesów na nią się składających oraz zaimplementowanie jako działającego programu rozwiązującego problemy pracy w grupie.
\\
Projekt ma zarówno wyjaśnić istniejący problem dotykający tego typu kooperacji między ludźmi a także dostarczyć gotowe rozwiązania niwelujące zaistniałe niedogodności oraz uprościć procesy powiązane z tym zagadnieniem. Głównym zadaniem jest próba poprawy mechanizmów komunikacji, systematyzacji pracy oraz podziału zadań między poszczególnych członków zespołu w sposób łatwy, wygodny a przede wszystkim efektywny, wydajny i przynoszący zyski.
\section{Problem}
Problemy wynikające z potrzeby współpracy z innymi ludźmi spotyka się zazwyczaj w sytuacjach, gdzie realizowany jest skomplikowany, obszerny projekt, wymagający znajomości wielu obszarów powiązanych ze sobą zagadnień, gdzie należy zastosować podział pracy na mniejsze fragmenty pomiędzy wielu ludzi, którzy posiadają odpowiednią wiedzę lub doświadczenie. Im większy, bardziej złożony projekt tym potrzeba podziału obowiązków jest silniejsza. Czy to wynikająca z samej struktury projektu czy innych czynników zewnętrznych takich jak ograniczenia czasowe lub jakościowe. \\
W dalszych następstwach powstają kolejne problemy, wynikające już z samej ilości zaangażowanych do projektu osób. Aby praca zespołowa przynosiła efekty należy musi być ona dobrze zorganizowana oraz zaplanowana.
\chapter{Analiza istniejących rozwiązań}  
\section{Badanie rynku}
\section{Zastosowane technologie}
\subsection{Struktura aplikacji}
Zaprojektowana aplikacja została podzielona na trzy główne części ze względu na cel oraz środowisko działania. Każda z nich została zaprojektowana w sposób autonomiczny, niezależny od pozostałych. Oznacza to, że można je traktować w sposób modularny, taki, który pozwala w bezinwazyjny dla aplikacji sposób zamienić je (oraz każdy z osobna) na inne rozwiązanie. Jedyną niezmienną rzeczą jest sposóB w jaki poszczególne części się ze sobą komunikują i wymieniają dane. 
\subsubsection{Warstwa frontendowa}
Warstwa odpowiadająca za obsługę części widocznej dla użytkownika aplikacji. Ma ona za zadanie obsługę zdarzeń wykonywanych przez użytkownika takich jak gesty myszki(kliknięcia, najeżdżanie kursorem, użycie klawiatury), przełączanie między podstronami a także wyświetlanie żądanej treści, jej układ na stronie, wygląd poszczególnych elementów, pozycja względem innych.
\\
Warstwa ta odpowiada także za wysyłanie żądań HTTP do zewnętrznych oraz wewnętrznych serwisów w celu pozyskania oraz przetworzenia otrzymanych danych. W skład zadań dotyczących tego zagadnienia wchodzi obsługa części błędów, mogących się pojawić podczas korzystania z aplikacji.
\subsubsection{Warstwa backendowa}
Warstwa ta zajmuje się częścią po stronie serwera, nie jest ona bezpośrednio widoczna dla użytkownika. Zadaniem tej warstwy jest nasłuchiwanie i odbieranie zapytań (np. HTTP) z zewnątrz, wliczając w to poprzednią warstwę frontendową. Zajmuje się ona także połączeniem z serwisem bazodanowym, odbieraniu oraz wysyłąniu danych w ramach przewarzania konkretnych zapytań (w zależności od zaimplementowanego systemu bazodanowego).
\\
Odbywają się tu takżę takie procesy jak uwierzytelnianie oraz podtrzymywanie sesji już zalogowanego użytkownika. Implementuje również obsługę błędów, głównie związanych z połączeniam bazodanowym.
\\
Warstwa ta ma za zadanie również serwować dane na zewnątrz aplikacji w formie API (Application Programming Interface - Interfejs Programowania Aplikacji)\cite{API}. Oznacza to, że momencie przejścia pod konkretny adres aplikacji (co jest jednoznacze z wysłaniem zapytania) jako informację zwrotną otrzymamy żądane dane w formacie JSON(JavaScript Object Notation)\cite {JSON}. Dane te można pote mw dowolny sposóB wykorzystać.
\\
Została tutaj zaimplementowana także forma zabezpieczenia niektórych warstw danych przed nieautoryzowanym dostępem.
\subsubsection{Warstwa systemu bazodanowego}
Warstwa systemu bazowego odpowiada za implementację oraz stworzenie fizycznej bazy danych na podstawie struktury danych. Do bazy danych łączy się warstwa backendowa posiadając odpowiednie dane autentykujące, takie jak nazwa bazy danych, adres serwera, nazwa oraz hasło użytkownika. Dane te są ustalane przez programistę.
\\
W odpowiedniku fizycznym tej warstwy są przechowywane wszelkie dane aplikacji.
\subsection{Języki programistyczne}
Warstwa frontendowa została napisana w języku JavaScript \cite {JS}, z wykorzystaniem standardu EcmaScript 6 \cite {ES6} oraz składni TypeScript \cite {TS}. Dodatkowo został użyty język znaczników HTML5 (Hype Text Markup Language) \cite {HTML}, który służy do reprezentacji treści, jaką przeglądarka internetowa jest w stanie interpretować oraz kaskadowych arkuszy styli CSS3 (Cascade Style Sheets) \cite {CSS}, dzięki którym możlliwym było stworzenie przejrzystego interfejsu. Użycie znaczników HTML 5 sprawiło na dodatek, że aplikacja jest przygotowana do wypozycjonowanie SEO(Search Engine Optimization) \cite {HTML_SEO} a wdrożenie najnowszych funkcjonalności CSS3, takich jak animacje oraz obsługa niektórych zdarzeń myszki zwiększyło szybkość ładowania się aplikacji, gdyż nie potrzebna jest implementacja po stronie języka JavaScript, który obciąża łącze internetowe oraz procesor \cite {JS_CPU}. Koejnym atutem jest fakt, iż nawet przy wyłączonej obsłudze JavaScript przeglądarki, aplikacja nadal będzie w pewien sposób używalna (chociażby po to, żeby móc poinformować użytkownika o konieczności włączenia obsługi JavaScript).
\\
\\
Warstwa backendowa również została zaimplementowana w języku JavaScript (z zasatosowaniem EcmaScript 6 oraz TypeScript. Posiadanie tego samego języka programowania w obu tych warstwach posiada szereg zalet, z których najważniejszymi jest czytelność kodu, poprzez podobne implementacje funkcjonalności na obu tych warstwach, co sprawia, że kod czytany przez osoby trzecie jest łatwiej i szybicej zrozumieć, co jest niezwykle ważne w dalszych procesach rozwojowych, w chwili gdy zajdzie potrzeba powiększenia zespołu programistycznego.
\\
Dodatkowym atutem jest możliwość użycia podobnych (jeśli nie tych samych) narzędzi programistycznych oraz konfiguracji środowiska.
\\
\\
System bazodanowy wykorzystuje implementację mongoDB \cite{MongoDB}, który to posiada strukturę bazy typu NoSQL \cite{NO_SQL}. System ten jest bardzo prosty do skomunikowania z ykorzystującą język JavaScript warstwę backendową. Baza danych mongoDB wykorzystuje notację JSON. Nie implementuje powiązań typowych dla np. MySQL, gdzie wsytępują klucze obce oraz relacje. Każda struktura danych jest z założęnia nie powiązana z żadną inną (w sensie implementacyjnym, po stronie samego systemu mongoDB). Sposób relacji zatem określa się dopiero po stronie programowej(tu: warstwa backendowa) i jest on dowolny, optymalny do potrzeb i sytuacji.

\subsection{Wykorzystane rozwiązania}
\subsubsection{React.js}
React.js \cite {React} został wykorzsytany w aplikacji jako podstawa warswt frontendowej. Wykorzystuje koncepcję tworzenia komponentów JavaScriptowych, dzięki czemu kod napisany jest bardzo modularny i reużywalny.
\subsubsection{Redux}
Redux \cite {Redux} jest rozszerzeniem do React.js, pozwalającym na lepszą organizację kodu poprzez automatyzację części funkcjonalności.
\subsubsection{Axios}
Axios \cite {Axios} to biblioteka JavaScriptowa, która została wykorzystana do wysyłania zapytań HTTP na serwer(warstwa backendowa)
\subsubsection{Socket.io}
Socket.io \cite {Socket.io} to biblioteka implementująca koncepcję Web Socket \cite {web_socket}. ZOstała użyta po stronie warstwy frontendowej oraz backendowej w celu zapewnienia komunikacji w czasie rzeczywistym podczas działania zaimplementowanego komunikatora. Rozwiązanie te pozwala na dosyć istotną redukcję ilości wysyłanych i odebranych zapytań w aplikacji poprzez obsługę tylko zdarzeń, które rzeczywiście miały miejsce, eliminuje to problem cyklicznego "odpytywania" warstwy backendowej w celu sprawdzenia czy baza danych została zmieniona.
\subsubsection{Sortable.js}
Bilbioteka Sortable.js \cite{Sortable} to obszerne narzędzie do zarządzania elementami typu przeciągnij-oraz-upuść(tzw. Drag and Drop). ZOstała zaimplementowana w celu podniesienia poziomu User Experience \cite {UX}.
\subsubsection{Moment.js}
Biblioteka ta \cite {Moment} służy do formatowania daty, odliczania czasu. Została zaimplementowana w miejscach gdzie nastąpiła referencja do daty stworzenia elementu.
\subsubsection{Node.js}
\subsubsection{Express}
\subsection{Wykorzystane narzędzia}
\chapter{Założenia projektowe}
\chapter{Wymagania dotyczące projektu}
\section{Wymagania funkcjonalne}
\begin{tabularx}{\textwidth}{ l | X | l | l }
\textbf {ID} & \textbf{Nazwa} & \textbf{Typ} & \textbf{Powiązania} \\
\hline
1 
& Rejestracja nowego konta
& Wymagane
& -
\end{tabularx}
\\ \\
\textbf{Przesłanka:} \\
Aby korzystać z funkcjonalności jakie oferuje aplikacja należy posiadać założone konto. Korzystanie z aplikacji wymaga uwierzytelnienia. Wiele funkcjonalności użytkowych musi być powiązane z inicjatorem zdarzenia / wykonanej akcji, takie jak przypisanie użytkownika jako autora do pewych stworzonych struktur.
\\ \\
\textbf{Interesariusze:} \\
Potencjalni użytkownicy aplikacji, użytkownicy nieposiadający zarejestrowanego konta w serwisie
\\ \\
\textbf{Opis:} \\
W celu umożliwienia korzystania z aplikacji nowym użytkownikom należy stworzyć formularz rejestracyjny na nowej podstronie z wymaganymi polami:
\begin{itemize}
	\item Nazwa nowego projektu
	\item Nazwa konta użytkownika
	\item Hasło użytkownika
	\item Potwierdzenie hasła użytkownika
\end{itemize}
\ \\
Wszystkie pola wymagają poprawnej walidacji pod kątem poprawności danych według następujących zasad:
\begin{itemize}
	\item Żadne pole nie może być puste
	\item Pole z nazwą projektu oraz nazwą użytkownika musi mieć conajmniej 6 znaków oraz być unikalne(inne od już stworzonych)
	\item Pole z hasłem użytkownika musi mieć conajmniej 7 znaków oraz zawierać conajmniej jedną cyfrę
	\item Pole z hasłem użytkownika oraz potwierdzeniem muszą być identyczne
\end{itemize}
\ \\
Podczas próby wysłania formularza z błędneie wypełnionymi polami powinna pojawić się zawierająca treść błędów informacja
\\
Gdy pola zostaną zweryfikowane(zwalidowane) poprawnie, po wysłaniu formularza zostanie wprowadzony nowy rekord do bazy danych zawierający dane nowego użytkownika a sam użytkownik zostanie automatycznie przekierowany na stronę logowania.
\section{Wymagania niefunkcjonalne}
\chapter{Projekt aplikacji}
\section{Zastosowane rozwiązania}
\section{Struktura danych}
\addcontentsline{toc}{chapter}{Bibliografia}
\bibliography{bibliografia}


\end{document}
